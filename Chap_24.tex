\documentclass[12pt]{article}

\usepackage[textheight=700pt, textwidth=500pt]{geometry}

\newcommand{\ind}[1][20pt]{\advance\leftskip + #1}
\newcommand{\deind}[1][20pt]{\advance\leftskip - #1}
\newenvironment{indt}[2][20pt]{#2 \par \ind[#1]}{\par \deind}

\usepackage{amssymb}
\usepackage{mathtools}
\usepackage{stmaryrd}
\usepackage{mathrsfs}
\usepackage{enumitem}

\setlength{\parindent}{0pt}
\setlength{\parskip}{5pt}
\pagenumbering{gobble}

\begin{document}
  \begin{indt}{\section*{Chapitre 24}}
    \begin{indt}{\subsection*{Prop 2.3}}
    
    
      Soit $f\in\mathcal{L}(E,F)$. Alors :
      
      \vspace{-10pt}
      
      \begin{enumerate}[leftmargin=2cm,rightmargin=2cm]
      \item $\mathrm{Ker}(f)$ est un sev de $E$;
      \item $\mathrm{Im}(f)$ est un sev de $F$.
      \end{enumerate}
      
      %\vspace{-10pt}
      
      \textbf{Démo :}
      
      %\vspace{-10pt}
      
      \begin{enumerate}[leftmargin=2cm,rightmargin=2cm]
      \item On a $0_E\in\mathrm{Ker}(f)$
      
      Donc $\mathrm{Ker}(f)\not =\emptyset$
      
      $\begin{array}{ll}
        \text{Si } & x,x'\in Kef(f)
        \\
        & \lambda, \lambda'\in K
      \end{array}$
      
      $\begin{array}{lll}
        \text{Alors }f(\lambda x+\lambda' x') & =\lambda f(x)+\lambda' f(x')
        \\
        & =\lambda 0 +\lambda' 0
        \\
        & =0
      \end{array}$
      
      Donc $\lambda x +\lambda' x'\in\mathrm{Ker}(f)$
      
      Donc $\mathrm{Ker}(f)$ est un sev de E.
      
      \item On a $0_F\in\mathrm{Im}(f)$
      
      Donc $\mathrm{Im}(f)\not =\emptyset$
      
      $\begin{array}{lll}
        \text{Soient} & y,y'\in\mathrm{Im}(f)
        \\
        & \lambda, \lambda'\in K
        \\
        & \begin{array}{ll}
          x,x'\in\mathrm{Ker}(f) \text{ tq} & y=f(x)
          \\
          & y'=f(x')
        \end{array}

      \end{array}$
      
      $\begin{array}{lll}
        \text{Alors }\lambda y+\lambda' y' & =\lambda f(x)+\lambda' f(x')
        \\
        & =f(\lambda x+\lambda' x')\in \mathrm{Im}(f)
      \end{array}$
      
      Donc $\mathrm{Im}(f)$ est un sev de F.
      \end{enumerate}
 
    \end{indt}

    \newpage
 
    \begin{indt}{\subsection*{Théo 2.4}}

      Soit $f\in\mathcal{L}(E,F)$. Alors :
      
      f injective $\Longleftrightarrow\mathrm{Ker}(f)=\{0_E\}$

      et

      f surjective $\Longleftrightarrow\mathrm{Im}(f)=F$

      \vspace{10pt}
      
      \textbf{Démo :}

      \vspace{10pt}

      \underline{f injective $\Longleftrightarrow\mathrm{Ker}(f)=\{0_E\}$}

      Si $\mathrm{Ker}(f)=\{0_E\}$

      Soient $x,x'\in E$ tq $f(x)-f(x')$

      Mque $x=x'$

      On a $f(x)-f(x')=0$

      Donc $f(x-x')=0$

      Donc $x-x'\in\mathrm{Ker}(f)=\{0\}$

      Donc $x-x'=0$

      \vspace{10pt}

      \textbf{Réciproque :}

      \vspace{10pt}

      Si $f$ est injective

      On sait que $0\in\mathrm{Ker}(f)$ (car $f$ est linéaire)

      Mais si $x\in\mathrm{Ker}(f)$, alors $f(x)=0=f(0)$

      Par injectivité, $x=0$ et $\mathrm{Ker}(f)=\{0\}$

      \vspace{10pt}

      \underline{f surjective $\Longleftrightarrow\mathrm{Im}(f)=F$}

      C'est général, rien à voir avec l'algé linéaire.

    \end{indt}

    \newpage

    \begin{indt}{\subsection*{Prop 2.6}}

      Soit $f\in\mathcal{L}(E)$. Alors $\mathrm{Im} (f^2)\subset\mathrm{Im} (f)$
      
      $\begin{array}{ll}
        \text{Alors } & \mathrm{Im}(f^2)\subset\mathrm{Im}(f)
        \\
        & \mathrm{Ker}(f)\subset\mathrm{Ker}(f^2)
      \end{array}$ 
      où $f^2=f\circ f$

      \vspace{10pt}

      \textbf{Démo :}

      \vspace{10pt}

      \underline{$\mathrm{Im}(f^2)\subset\mathrm{Im}(f)$}

      Soit $y\in\mathrm{Im}(f^2)$
        
      $\begin{array}{ll}
      \text{Alors }y & =f^2(x)
      \\
      & =f(f(x))
      \end{array}$
      où $x\in E$

      Donc $y=\mathrm{Im}(f)$

      \vspace{10pt}

      \underline{$\mathrm{Ker}(f)\subset\mathrm{Ker}(f^2)$}

      Soit $x\in\mathrm{Ker}(f)$

      $\begin{array}{lll}
        \text{Alors } f^2 & =f(f(x))
        \\
        & =f(0)
        \\
        & =0
      \end{array}$
      où f linéaire

      Donc $x\in\mathrm{Ker}(f^2)$

    \end{indt}
  \end{indt}
\end{document}
